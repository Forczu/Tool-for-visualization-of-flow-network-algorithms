% !TeX spellcheck = pl_PL
\chapter{Podsumowanie}
Celem niniejszej pracy było zrealizowanie wizualizacji działań na abstrakcyjnym modelu danych jakim są sieci przepływowe. Skupiono się na prezentacji algorytmów rozwiązujących problem wyszukiwania maksymalnego przepływu, jednak nie jest to jedyna rodzina algorytmów jaką można realizować dzięki temu modelowi. Sieci przepływowe są skomplikowaną strukturą danych, o wiele bardziej złożoną niż zwykłe grafy. Problem maksymalnego przepływu jest najbardziej praktycznym i intuicyjnym jaki można rozwiązać dzięki sieci przepływowej, dlatego jest szeroko opisywany w literaturze. Mimo to, wymaga wielu dodatkowych obliczeń i tworzenia struktur pośrednich, co nie jest wymagane w klasycznych algorytmach operujących na prostych grafach, m.in. znajdowaniu najkrótszej ścieżki między wierzchołkami.\\\indent
Okazało się niemożliwym opisanej całej funkcjonalności oraz szczegółów implementacji jaką posiada aplikacja (m.in. sposobu łączenia rysunków wierzchołków i łuków w jedną całość), więc w specyfikacji wewnętrznej skupiono się na opisaniu najistotniejszych elementów oprogramowania, które były bezpośrednio związane z tematem pracy inżynierskiej, m.in. reprezentacji sieci w pamięci komputera, zapewnieniu jej zgodności z teorią oraz sposobach realizacji algorytmów na niej. Ponadto przedstawiono zastosowane wzorce projektowe oraz możliwości rozszerzania aplikacji, pomijając elementy związane z platformą Qt i implementacją okna aplikacji.\\\indent
Praca inżynierska była największym projektem programistycznym w trakcie studiów. Wymagała całego zakresu wiedzy, jaki pojawił się na programowaniu komputerów i inżynierii oprogramowania, a także wiele dodatkowego samokształcenia. Największym wyzwaniem było poprawne zaprojektowanie architektury aplikacji, która decydowała o całym późniejszym kierunku rozwoju. Innym kształcącym problemem było przeniesienie abstrakcyjnej struktury danych do pamięci komputera zgodnie z teorią zawartą w literaturze. Wyzwaniem było także zachowanie dobrych praktyk programistycznych. Zdarzały się problemy, które zostały rozwiązane w nieelegancki problem, a później powróciły w trakcie pisania kolejnych funkcjonalności. Dzięki temu wyciągnięto wiele wiedzy i doświadczenia jakie płyną z pracy nad dużym projektem.\\\indent
Aplikacja spełnia wszystkie założenia i wymagania jakie były podane w temacie pracy inżynierskiej. Wykonano także dużo dodatkowej funkcjonalności z myślą o wygodzie użytkownika i wiedzy, jaką powinien wynieść korzystając z tego programu. Aplikacja może być bez problemów wykorzystywana jako pomoc dydaktyczna w trakcie zajęć z algorytmiki. Wszystkie przedstawione algorytmy wyznaczania maksymalnego przepływu wykonuje poprawnie, prezentuje w jaki sposób są wykonywane krok po kroku i ilustruje wszystkie operacje. Dzięki niej studentowi, który musi przyswoić sieci przepływowe, na pewno uda się to łatwiej i szybciej. Aplikacja jest bezpieczna, nie wykonuje algorytmów na źle zbudowanych sieci i informuje użytkownika co należy poprawić, co również jest praktyczną pomocą przy nauce tych struktur danych. Na rynku nie ma wiele gotowych aplikacji edukacyjnych, ani nawet zaawansowanych bibliotek dedykowanych grafom, więc utworzony program stanowi świetne rozwiązanie dla każdego, kto chciałby nauczyć się więcej o sieciach przepływowych. Dzięki zastosowanym wzorcom projektowym oraz hierarchii klas aplikację można rozwijać dalej, wzbogacając ją w przyszłości o operowanie na większej liczbie rodzajów grafów i kolejnych rodzinach algorytmów.